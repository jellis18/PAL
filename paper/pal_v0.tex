\documentclass[prd,onecolumn]{revtex4}


\usepackage{amsmath,amssymb}
\usepackage{multirow}
\usepackage{bbold}
\usepackage{graphicx}
\usepackage{subfigure}
\newcommand{\be}{\begin{equation}}
\newcommand{\ee}{\end{equation}}
\newcommand{\omhat}{\hat{\Omega}}
\newcommand{\phat}{\hat{p}}
\newcommand{\hplus}{h_+}
\newcommand{\hcross}{h_{\times}}
\newcommand{\infint}{\int_{-\infty}^{\infty}}
\newcommand{\lp}{\left(}
\newcommand{\rp}{\right)}
\newcommand{\bb}{\begin{bmatrix}}
\newcommand{\eb}{\end{bmatrix}}
\DeclareMathOperator{\Tr}{Tr}


\begin{document}


\title{PAL: The Pulsar timing array Algorithm Library }


\author{J. Ellis}
\affiliation{Department of Physics, University of Wisconsin-Milwaukee,
 Milwaukee, WI 53201, USA}
 
\begin{abstract}

The effort to detect Gravitational Waves (GWs) with Pulsar Timing Arrays (PTAs) is a very vibrant and growing field in astrophysics. Recently there have been many new data analysis techniques introduced aimed at detection and characterization of GWs and the general noise properties of pulsars, however; there has been no development of a central \emph{open source} code base to house these different techniques. The Pulsar timing array Algorithm Library (PAL) seeks to fulfill this goal. In this manuscript we will review the algorithms currently present in PAL and introduce the overall structure and goal of this \emph{collaborative} project. 

\end{abstract}

\maketitle

\section{Introduction}

\section{General Overview of Algorithms}

\section{Interface to tempo2}

\section{PALDetect}

\section{PALSimulation}

\section{PALInference}

\section{Post Processing Tools}

\section{API}




\end{document}
